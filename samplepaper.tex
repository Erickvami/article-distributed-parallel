% This is samplepaper.tex, a sample chapter demonstrating the
% LLNCS macro package for Springer Computer Science proceedings;
% Version 2.20 of 2017/10/04
%
\documentclass[runningheads]{llncs}
%
\usepackage{graphicx}
% Used for displaying a sample figure. If possible, figure files should
% be included in EPS format.
%
% If you use the hyperref package, please uncomment the following line
% to display URLs in blue roman font according to Springer's eBook style:
% \renewcommand\UrlFont{\color{blue}\rmfamily}

\begin{document}
%
\title{An Event-based Architecture for Multi-population Optimization Algorithms\thanks{Supported by organization x.}}
%
%\titlerunning{Abbreviated paper title}
% If the paper title is too long for the running head, you can set
% an abbreviated paper title here
%
\author{Erick Vargas Minguela\inst{1}\orcidID{0000-1111-2222-3333} \and
Mario Garcia Valdez\inst{2,3}\orcidID{1111-2222-3333-4444} \and
Third Author\inst{3}\orcidID{2222--3333-4444-5555}}
%
\authorrunning{F. Author et al.}
% First names are abbreviated in the running head.
% If there are more than two authors, 'et al.' is used.
%
\institute{National Technological Institute of Mexico 
\email{erick.vargas.minguela@gmail.com}\\
\url{http://www.springer.com/gp/computer-science/lncs}\\
\email{\{abc,lncs\}}}
%
\maketitle              % typeset the header of the contribution
%
\begin{abstract}
    Having the knowledge that both of them are population-based algorithms it can be
    defined that a migration between 2 or more populations are possible, and this kind
    hybrid can be helpful to increase the possibility to find the optimal result (the best
    of the best), there is where fits the concept of Multi-population.
    For this kind of work we used asynchronous functions, serverless functions,
multithread and a distributed architecture taking advantage for functional
programming and serverless architecture.
Even nature works like that... parallel, asynchronous and distributed.



\keywords{Multi-population  \and Asynchronous \and Sub-population \and Serverless \and Distributed.}
\end{abstract}
%
%
%
\section{Introduction}
A universe of solutions can exist for a single problem and sometimes is too big and complex 
to solve them in a traditional way. That is why heuristic and population based algorithms are required. This kind
of algorithms are very useful to solve combinatories problems, however, usually one is better than others 
to solve one thing and another is great to solve another problem, and there are several cases 
stucked in optimal local values.

Here we propose an architecture composed by serverless functions to create a multi-populations that
will process sub-populations in distributed function architecture and they are going to be parallel and asynchronous
making each sub-population distributed and independent. And comunicating using migrations to help each other 
preventing fall into optimal local values.

The distributed architectures are having extensive use in the software
industry because of their high performance, many systems are being
created and migrating step by step to microservices and... in a nearly
future... the new architectures called serverless, which proposes the
use of “Function as a Service” (FaaS).


\subsection{Serverless}
Recently, the cloud providers as Amazon Web Services (AWS), Google Cloud, etc. Offers a new alternative to programming throught interfaces called Serverless Computing,
this kind of platform consist in a very simple mecanism where the developer upload the code into the platform and 
execute it as mamy times it is required scalling and allowing do this in a parallel way. This way the developers do not worry about servers, connections and other configurations.
In serverless it pays only for what is used. Even there are some platform that allows to install them into your own server to do your own local architecture serverless.

\begin{figure}[htp]
  \includegraphics[width=\textwidth]{architectures.png}
  \caption{Software architecture generations.} \label{fig1}
  \end{figure}

\subsubsection{Serverless Function} 
In math a function is a relation between a set of inputs and an allowed set of outputs with
the idea that each input goes to a single output. But in computer science is small bits of code 
that do only one thing and are easily to understand
and support. In serverless this functions could be triggered by an event that would be menssages, http request,etc.
Also is known that each function scales independently and is stateless with a short duration.


\section{Proposed architecture}

The propose is an architecture that allows to process a simple population that is divided into sub-populations
distributing them to process in different ways and then communicating each other with purpose of increase the possibility
to find the best fitness of a function. This architecture can accept the use of an indeterminate number of algorithms, allowing
an easy hybridation and continuous adaptability for different problems.

This architecture consist in 3 nodes, they are explained on the next points:

\begin{itemize}
  \item Manager: Here the multi-population is managed, it is initialized how to process it, 
  followed by the subpopulations, which when they are concepted trigger an event that sends 
  the subpopulations to be stored in a file JSON of MongoDB (preventing to saturate the memory) 
  and to a message queue that is directed to its subsequent processing in the “Receiver” section 
  that is our cluster of functions (Faas), because each subpopulation requires the execution of a 
  different algorithm, there is a different channel in the web sockets for each type of algorithm 
  that triggers its respective serverless function.
  Once a subpopulation is processed, it is returned and a selection is made for the sub-population 
  migration. The migration selection is made by taking the population attribute of the subpopulation 
  that was returned and the subpopulation that it have been selected among the best 2, it should be 
  noted that the decision to identify the best from the 2 is made randomly. Once the selection is made, 
  a Splitting Point Uniform crossing will be made.
  The 2 new subpopulations replace their self and are resent it back to their respective
serverless function and this process is repeited until completing the number of assigned migrations
for the multi-population. Of course this whole process is performed asynchronously avoiding wait for all responses from serverless functions to perform a crossover or a
update of the multi-population status \cite{Lovbjerg2001,Jimeno2019}. In the following figure you can see from
illustrative way how the multi-population is composed.
\begin{figure}[htp]
  \centering
  \includegraphics[width=0.7\textwidth]{multipopulation.png}
  \caption{Multipopulation representation.} \label{fig1}
  \end{figure}
\item Message Provider: Its purpose is the creation of a sub-population messages queue which is the 
comunication channel between the sub-population Manager and the Receiver (FaaS), each message is a 
trigger for the execution of a GA or PSO function. Thanks to the message queue, it is possible to perform 
the serverless functions asynchronously, avoiding that the algorithm wait for responses independently of 
their duration and the simultaneous evaluation of different sub-populations independent of its algorithm 
or characteristics.
\item Receiver: The following section contains the Serverless functions of the algorithms to be executed, 
reduced the best possible using the functional programming paradigm so that they can be converted into FaaS 
without problems, in addition to achieving a completely clean and fast execution \cite{Roberts2016} . Each message received 
on this node is executed in the form of a multi-threaded process in parallel, this allows to having more than 
one population-based search algorithm running at the same time, and making a copy of itself each algorithm 
function as required.
\end{itemize}


To develop this architecture the applied technologies are based in JavaScript using Node JS as it can be
see it in the General Architecture Flowchart.

\begin{figure}[htp]
  \centering
  \includegraphics[width=0.9\textwidth]{general diagram architecture 2.png}
  \caption{General Architecture Flowchart.} \label{fig1}
  \end{figure}

\subsection{Sub-population definition}

Individuals are created composed of 2 types of information, the one that is active or useful for crossing
and the one that is not. The population contains the series of possible solutions, while the rest contains 
the information on how the processing for the search of the optimal solutions will be executed, linked directly 
with their respective algorithms.

\begin{figure}[htp]
  \centering
  \includegraphics[width=0.6\textwidth]{subpopulationDefinition.png}
  \caption{Sub-population composition.} \label{fig1}
  \end{figure}

\subsection{Splitting Point Uniform Migration}
A uniform mask is created to apply the migration between individuals
from 2 sub-populations. The selected data are combined using the middle point between the
selected points by the mask. This process iterates the 2 sub-populations to randomly swap values and
replace some of them with middle points.


\begin{figure}[htp]
  \centering
  \includegraphics[width=0.9\textwidth]{splittinPointUniform.png}
  \caption{Splitting Point Uniform process.} \label{fig1}
  \end{figure}

  \subsection{Migration Selection}

  Using migration selection by tournament keeping up the information of the best sub-population of the multi-population.

\begin{figure}[htp]
  \centering
  \includegraphics[width=0.5\textwidth]{selection.png}
  \caption{Selection by tournament.} \label{fig1}
  \end{figure}


  % \begin{theorem}
  %   This is a sample theorem. The run-in heading is set in bold, while
  %   the following text appears in italics. Definitions, lemmas,
  %   propositions, and corollaries are styled the same way.
  %   \end{theorem}
  %   %
  %   % the environments 'definition', 'lemma', 'proposition', 'corollary',
  %   % 'remark', and 'example' are defined in the LLNCS documentclass as well.
  %   %
  %   \begin{proof}
  %   Proofs, examples, and remarks have the initial word in italics,
  %   while the following text appears in normal font.
  %   \end{proof}
% articles~\cite{ref_article1}, an LNCS chapter~\cite{ref_lncs1}, a
% book~\cite{ref_book1}, proceedings without editors~\cite{ref_proc1},
% and a homepage~\cite{ref_url1}. Multiple citations are grouped
% \cite{ref_article1,ref_lncs1,ref_book1},
% \cite{ref_article1,ref_book1,ref_proc1,ref_url1}.

%-------------------Experiments-----------------------------------------------

\section{Experiments and Results}
\subsection{Experiments}
Now that an interaction between sub-populations with different algorithms it is working and  hybridation have been a success, using until now the added algorithms (GA and PSO) algorithms, all thanks to the 
developed architecture, lets procede to the experiments.
This section is going to be the execution of several experiments from 2 to 40 dimensions, with a stop criterial 
of an error below  0.5E-8, without a parameter optimization method, waiting that the architecture by its self
would be enough to increase the possibility to find a better optimal result than the traditional methods.
All this hoping that the results will probe the needness of this kind of architecture on increasing dimensions.
To test if the architecture was useful, several experiments were made to solve
benchmark functions, for this case the functions are Sphere, Rastrigin and Rosenbrock.
Using 10 sub-population for each experiment and maximum 4 migrations per sub-population with different algorithms and parameters for each sub-population.

\begin{figure}[htp]
  \centering
    \includegraphics[width=0.7\textwidth]{benchmark.png}
    \caption{Benchmark functions for experimentation.} \label{fig1}
    \end{figure}

\subsection{Parameters Configuration} 

This architecture modifies the traditional way to work with population based algorithms, then the experiments 
could not be parameterized as usually are.

Then the experiments are scaled by their number of evaluations and the
parameters must be configured to be adjusted to the next criterial, using the next expression:

\begin{equation}
    \label{eq:hesitancy-interpretation}
   Evaluations = 10^{5} Dimensions
   \end{equation}

For example, if the experiment has 2 dimensions, the maximum number of evaluations will be 200,0000, for 10 dimensions will be 1,000,000 of evaluacions and the same with the others dimensions.









   \begin{table}[htp]
    \caption{2 dimension parameters}
    \label{table:ga-pso-parameters-2}
    \centering
    \begin{tabular}{|l|l|}
    \hline
    Parameter & Value \\
    \hline
    \hline
    GA Optimization & Minimize \\
    \hline
    GA Generations & 50 \\
    \hline
    GA Dimensions & 2 \\
    \hline
    GA Population size & 100 \\
    \hline
    GA Mutation & Random(Tournament2,Tournament3,Random \\
    &  ,RandomLinearRank,Sequential,Fittest)\\
    \hline
    GA Crossover & Tournament3 \\
    \hline
    GA Crossover percentage & Random[10\%, 80\%] \\
    \hline
    GA Mutation percentage & Random[10\%,50\%] \\
    \hline
    GA Crossover function & Uniforme de punto medio \\
    \hline
    GA Mutation Function & gaussian \\
    \hline
    PSO Optimization & Minimiza \\
    \hline
    PSO Iterations & 50 \\
    \hline
    PSO Dimensions & 2 \\
    \hline
    PSO Vector size & 100 \\
    \hline
    PSO Social factor & Random[0.5,4.0] \\
    \hline
    PSO Individual factor & Random[0.5,4.0] \\
    \hline
    PSO Inercia factor & Random[0.5,4.0] \\
    \hline
    \end{tabular}
    \end{table}

\begin{figure}[htp]
\includegraphics[width=\textwidth]{2-sphere.jpg}
\caption{2 dimension experiments Sphere.} \label{fig1}
\end{figure}

\begin{figure}[htp]
  \includegraphics[width=\textwidth]{2-rastrigin.jpg}
  \caption{2 dimension experiments Rastrigin.} \label{fig1}
  \end{figure}
  
\begin{figure}[htp]
  \includegraphics[width=\textwidth]{2-rosenbrock.jpg}
  \caption{2 dimension experiments Rosenbrock.} \label{fig1}
  \end{figure}

\begin{table}[htp]
    \caption{2 dimensional experiment results}
    \label{table:resultados-2}
    \centering
    \begin{tabular}{|c|c|c|c|}
    \hline
    Fn & Best & AVG & Experiment Number \\
    \hline
    \hline
    Rastrigin GA & 0 & 1.65377E-08 & 15\\
    \hline
    Rastrigin PSO & 0 & 1.8872E-12 & 15\\
    \hline
    Rastrigin GA-PSO & 0 & 0 & 15\\
    \hline
    Sphere GA & 4.53222E-18 & 4.36977E-10 & 15\\
    \hline
    Sphere PSO & 0 & 7.8012E-12 & 15\\
    \hline
    Sphere GA-PSO & 0 & 4.33161E-14 & 15\\
    \hline
    Rosenbrock GA & 1.62335E-13 & 1.24176E-08 & 15\\
    \hline
    Rosenbrock PSO & 1.11674E-12 & 2.47795E-06 & 15\\
    \hline
    Rosenbrock GA-PSO & 9.5809E-14 & 6.90695E-09 & 15\\
    \hline
    \end{tabular}
    \end{table}
  
    \begin{table}[htp]
      \caption{10 dimensions parameters}
      \label{table:ga-pso-parameters-10}
      \centering
      \begin{tabular}{|l|l|}
      \hline
      Parameter & Value \\
      \hline
      \hline
      GA Optimization & Minimiza \\
      \hline
GA Generations & 70 \\
      \hline
GA Dimensions & 10 \\
      \hline
GA Population size & 200 \\
      \hline
GA Mutation & Random(Tournament2,Tournament3,Random \\
      &  ,RandomLinearRank,Sequential,Fittest)\\
      \hline
GA Crossover \\
      \hline
GA Crossover percentage & Random[10\%, 80\%] \\
      \hline
GA Mutation percentage & Random[10\%,50\%] \\
      \hline
GA Crossover function & Uniforme de punto medio \\
      \hline
GA Mutation Function & gaussian \\
      \hline
PSO Optimization & Minimiza \\
      \hline
PSO Iterations & 70 \\
      \hline
PSO Dimensions & 10 \\
      \hline
PSO Vector size & 200 \\
      \hline
PSO Social factor & Random[0.5,4.0] \\
      \hline
PSO Individual factor & Random[0.5,4.0] \\
      \hline
PSO Inercia factor & Random[0.5,4.0] \\
      \hline
      \end{tabular}
      \end{table}
    
      \begin{figure}[htp]
        \includegraphics[width=\textwidth]{10-sphere.jpg}
        \caption{10 dimensions experiments Sphere.} \label{fig1}
        \end{figure}

        \begin{figure}[htp]
          \includegraphics[width=\textwidth]{10-rastrigin.jpg}
          \caption{10 dimensions experiments Rastrigin.} \label{fig1}
          \end{figure}

          \begin{figure}[htp]
            \includegraphics[width=\textwidth]{10-rosenbrock.jpg}
            \caption{10 dimensions experiments Rosenbrock.} \label{fig1}
            \end{figure}

        \begin{table}[htp]
          \caption{10 dimensional experiment results}
          \label{table:resultados-2}
          \centering
          \begin{tabular}{|l|l|l|l|}
          \hline
          Fn & Best & Average & Experiment Number \\
          \hline
          \hline
          Rastrigin GA & 3.21768E-09 & 2.38015E-06 & 15\\
          \hline
          Rastrigin PSO & 7.8586E-11 & 2.715716161 & 15\\
          \hline
          Rastrigin GA-PSO & 8.01492E-12 & 5.08668E-09 & 15\\
          \hline
          Sphere GA & 1.84051E-09 & 2.5389E-08 & 15\\
          \hline
          Sphere PSO & 4.50351E-11 & 4.72855E-09 & 15\\
          \hline
          Sphere GA-PSO & 3.33851E-11 & 1.30062E-09 & 15\\
          \hline
          Rosenbrock GA & 9.58323E-07 & 1.24176E-08 & 15\\
          \hline
          Rosenbrock PSO & 4.16711E-07 & 4.431565444 & 15\\
          \hline
          Rosenbrock GA-PSO & 3.62472E-07 & 0.000240251 & 15\\
          \hline
          \end{tabular}
          \end{table}

          \begin{table}[htp]
            \caption{Parametros experimentos 20 dimensiones}
            \label{table:ga-pso-parameters-20}
            \centering
            \begin{tabular}{|l|l|}
            \hline
            Parameter & Value \\
            \hline
            \hline
            GA Optimization & Minimiza \\
            \hline
            GA Generations & 70 \\
            \hline
            GA Dimensions & 20 \\
            \hline
            GA Population size & 200 \\
            \hline
            GA Mutation & Random(Tournament2,Tournament3,Random \\
            &  ,RandomLinearRank,Sequential,Fittest)\\
            \hline
            GA Crossover & Tournament3 \\
            \hline
            GA Crossover percentage & Random[10\%, 80\%] \\
            \hline
            GA Mutation percentage & Random[10\%,50\%] \\
            \hline
            GA Crossover function & Uniforme de punto medio \\
            \hline
            GA Mutation Function & gaussian \\
            \hline
            PSO Optimization & Minimiza \\
            \hline
            PSO Iterations & 70 \\
            \hline
            PSO Dimensions & 20 \\
            \hline
            PSO Vector size & 200 \\
            \hline
            PSO Social factor & Random[0.5,4.0] \\
            \hline
            PSO Individual factor & Random[0.5,4.0] \\
            \hline
            PSO Inercia factor & Random[0.5,4.0] \\
            \hline
            \end{tabular}
            \end{table}
          
            \begin{figure}[htp]
              \includegraphics[width=\textwidth]{20-sphere.jpg}
              \caption{20 dimensions experiments Sphere.} \label{fig1}
              \end{figure}
      
              \begin{figure}[htp]
                \includegraphics[width=\textwidth]{20-rastrigin.jpg}
                \caption{20 dimensions experiments Rastrigin.} \label{fig1}
                \end{figure}
      
                \begin{figure}[htp]
                  \includegraphics[width=\textwidth]{20-rosenbrock.jpg}
                  \caption{20 dimensions experiments Rosenbrock.} \label{fig1}
                  \end{figure}

                  \begin{table}[htp]

    \caption{Resultados 20 dimensiones}
    \label{table:resultados-2}
    \centering
    \begin{tabular}{|l|l|l|l|}
    \hline
    Fn & Best & Average & Experiment Number \\
    \hline
    \hline
    Rastrigin GA & 0.000808633 & 0.220596203 & 15\\
    \hline
    Rastrigin PSO & 3.988070734 & 25.51777514 & 15\\
    \hline
    Rastrigin GA-PSO & 9.13E-09 & 7.38E-02 & 15\\
    \hline
    Sphere GA & 1.84051E-09 & 9.22715E-06 & 15\\
    \hline
    Sphere PSO & 7.04E-11 & 3.50E-07 & 15\\
    \hline
    Sphere GA-PSO & 9.11E-11 & 2.13E-08 & 15\\
    \hline
    Rosenbrock GA & 0.000348015 & 0.010958941 & 15\\
    \hline
    Rosenbrock PSO & 9.119539342 & 13.37613983 & 15\\
    \hline
    Rosenbrock GA-PSO & 2.31663E-05 & 0.005608855 & 15\\
    \hline
    \end{tabular}
    \end{table}

    \begin{table}[htp]
      \caption{Parametros experimentos 40 dimensiones}
      \label{table:ga-pso-parameters-20}
      \centering
      \begin{tabular}{|c|c|}
      \hline
      Parameter & Value \\
      \hline
      \hline
      GA Optimization& Minimiza \\
      \hline
      GA Generations & 70 \\
      \hline
      GA Dimensions & 40 \\
      \hline
      GA Population size & 200 \\
      \hline
      GA Mutation & Random(Tournament2,Tournament3,Random \\
      &  ,RandomLinearRank,Sequential,Fittest)\\
      \hline
      GA Crossover \\
      \hline
      GA Crossover percentage & Random[10\%, 80\%] \\
      \hline
      GA Mutation percentage & Random[10\%,50\%] \\
      \hline
      GA Crossover function & Uniforme de punto medio \\
      \hline
      GA Mutation Function & gaussian \\
      \hline
      PSO Optimization & Minimiza \\
      \hline
      PSO Iterations & 70 \\
      \hline
      PSO Dimensions & 40 \\
      \hline
      PSO Vector size& 200 \\
      \hline
      PSO Social factor & Random[0.5,4.0] \\
      \hline
      PSO Individual factor & Random[0.5,4.0] \\
      \hline
      PSO Inercia factor & Random[0.5,4.0] \\
      \hline
      \end{tabular}
      \end{table}
    
      \begin{figure}[htp]
        \includegraphics[width=\textwidth]{40-sphere.jpg}
        \caption{40 dimensions experiments Sphere.} \label{fig1}
        \end{figure}

        \begin{figure}[htp]
          \includegraphics[width=\textwidth]{40-rastrigin.jpg}
          \caption{40 dimensions experiments Rastrigin.} \label{fig1}
          \end{figure}

          \begin{figure}[htp]
            \includegraphics[width=\textwidth]{40-rosenbrock.jpg}
            \caption{40 dimensions experiments Rosenbrock.} \label{fig1}
            \end{figure}

            \begin{table}[htp]
              \caption{Resultados 40 dimensiones}
              \label{table:resultados-2}
              \centering
              \begin{tabular}{|l|l|l|l|}
              \hline
              Fn & Best & Average & Experiment Number \\
              \hline
              \hline
              Rastrigin GA & 1.95478879 & 3.560837088 & 15\\
              \hline
              Rastrigin PSO & 29.06596132 & 130.2865863 & 15\\
              \hline
              Rastrigin GA-PSO & 2.46E-04 & 2.13E+00 & 15\\
              \hline
              Sphere GA & 0.002686956 & 0.005302951 & 15\\
              \hline
              Sphere PSO & 8.68E-11 & 2.07E-03 & 15\\
              \hline
              Sphere GA-PSO & 2.00E-10 & 1.41E-04 & 15\\
              \hline
              Rosenbrock GA & 0.000348015 & 106.9287542 & 15\\
              \hline
              Rosenbrock PSO & 0.032708559 & 0.368395353 & 15\\
              \hline
              Rosenbrock GA-PSO & 0.018538924 & 0.525086565 & 15\\
              \hline
              \end{tabular}
              \end{table}
% ---- Bibliography ----
%
% BibTeX users should specify bibliography style 'splncs04'.
% References will then be sorted and formatted in the correct style.
%
% \bibliographystyle{splncs04}
% \bibliography{mybibliography}
%
\section{Conclusion}

This architecture is completely scalable and useful for hybridation of multiple algorithms,
until now is only GA and PSO but according with the results with this kind of architecture
there is no limit and it works better than multi-populations with only one algorithm, also
every experiment executes in short times because serverless functions searching in an asynchronous way getting a fast convergence.
   \begin{thebibliography}{10}

    \bibitem{Hellerstein2018}
    J.~M. Hellerstein, J.~Faleiro, J.~E. Gonzalez, J.~Schleier-Smith, V.~Sreekanti,
      A.~Tumanov, and C.~Wu, ``{Serverless Computing: One Step Forward, Two Steps
      Back},'' vol.~3, 2018.
    
    \bibitem{Kramer2017}
    O.~Kramer, ``{Genetic Algorithm Essentials},'' {\em Springer International
      Publishing AG}, vol.~679, pp.~11--20, 2017.
    
    \bibitem{Guerrero2017}
    C.~Guerrero, I.~Lera, and C.~Juiz, ``{Genetic Algorithm for Multi-Objective
      Optimization of Container Allocation in Cloud Architecture},'' {\em Journal
      of Grid Computing}, pp.~1--23, 2017.
    
    \bibitem{Lalwani2019}
    S.~Lalwani, H.~Sharma, S.~Chandra, S.~Kusum, D.~Jagdish, and C.~Bansal,
      ``{REVIEW - COMPUTER ENGINEERING AND COMPUTER SCIENCE A Survey on Parallel
      Particle Swarm Optimization Algorithms},'' {\em Arabian Journal for Science
      and Engineering}, 2019.
    
    \bibitem{Blum2005}
    S.~Blum, R.~Puisa, and M.~Wintermantel, ``{Adaptive Mutation Strategies for
      Evolutionary Algorithms},'' {\em 2nd Weimar Optimization and Stochastic
      Days}, pp.~1--13, 2005.
    
    \bibitem{Everywhere}
    S.~Everywhere, ``{The Fn Project},''
    
    \bibitem{Ma2019}
    H.~Ma, S.~Shen, M.~Yu, Z.~Yang, M.~Fei, and H.~Zhou, ``{Multi-population
      techniques in nature inspired optimization algorithms : A comprehensive
      survey},'' {\em Swarm and Evolutionary Computation}, vol.~44, no.~July 2017,
      pp.~365--387, 2019.
    
    \bibitem{Santander-jim2018}
    S.~Santander-jim and M.~A. Vega-rodr, ``{Comparative Analysis of
      Intra-Algorithm Parallel Multiobjective Evolutionary Algorithms : Taxonomy
      Implications on Bioinformatics Scenarios},'' vol.~9219, no.~c, pp.~1--15,
      2018.
    
    \bibitem{Sherry2012}
    D.~Sherry, K.~Veeramachaneni, J.~McDermott, and U.~M. O'Reilly, ``{Flex-GP:
      Genetic programming on the cloud},'' {\em Lecture Notes in Computer Science
      (including subseries Lecture Notes in Artificial Intelligence and Lecture
      Notes in Bioinformatics)}, vol.~7248 LNCS, pp.~477--486, 2012.
    
    \bibitem{Goebel2016}
    R.~Goebel, {\em {29th Australasian Joint Conference on Artificial Intelligence,
      AI 2016}}, vol.~9992 LNAI.
    \newblock 2016.
    
    \bibitem{Guerv2018}
    J.~J.~M. Guerv and J.~M. Garc, ``{Introducing an Event-Based Architecture for
      Concurrent and Distributed Evolutionary Algorithms},'' vol.~1, pp.~399--410,
      2018.
    
    \bibitem{Moroney2017}
    L.~Moroney, {\em {The Definitive Guide to Firebase: Build Android Apps on
      Google's Mobile Platform}}.
    \newblock 2017.
    
    \bibitem{Ambler2015}
    T.~Ambler and N.~Cloud, {\em {JavaScript frameworks for modern web dev}}.
    \newblock 2015.
    
    \bibitem{Barwell2016}
    A.~D. Barwell, C.~Brown, and K.~Hammond, ``{USING PROGRAM SHAPING AND
      ALGORITHMIC SKELETONS TO PARALLELISE AN EVOLUTIONARY MULTI-AGENT SYSTEM IN
      ERLANG Wojciech Turek , Aleksander Byrski},'' vol.~35, pp.~792--818, 2016.
    
    \bibitem{Paper2017}
    C.~Paper and E.~Alba, ``{Distributed Genetic Algorithms on Portable Devices for
      Smart Cities},'' no.~May, 2017.
    
    \bibitem{Technische2016}
    J.~L. Technische, ``{C6.3 Island (migration) models: evolutionary algorithms
      based on punctuated equilibria},'' no.~January 2000, 2016.
    
    \bibitem{Kunasaikaran2016}
    J.~Kunasaikaran and A.~Iqbal, ``{A Brief Overview of Functional Programming
      Languages},'' {\em electronic Journal of Computer Science and Information
      Technology (eJCSIT)}, vol.~6, no.~1, pp.~32--36, 2016.
    
    \bibitem{Hows2014}
    D.~Hows, P.~Membrey, and E.~Plugge, ``{MongoDB Basics},'' {\em MongoDB Basics},
      2014.
    
    \bibitem{Cook2017}
    J.~Cook, {\em {Docker for Data Science}}.
    \newblock 2017.
    
    \bibitem{Lovbjerg2001}
    M.~L{\o}vbjerg and T.~K. Rasmussen, ``{Hybrid Particle Swarm Optimiser with
      Breeding and Subpopulations},'' {\em Proc. 3rd Genetic Evolutionary
      Computation Conf.}, pp.~469 --476, 2001.
    
    \bibitem{Jimeno2019}
    H.~M.~A. Jimeno, M.~J.~L. S{\'{a}}nchez, and R.~H. Rico, ``{Multipopulation ‑
      based multi ‑ level parallel enhanced Jaya algorithms},'' {\em The Journal
      of Supercomputing}, no.~0123456789, 2019.
    
    \bibitem{Roberts2016}
    M.~Roberts, ``{Serverless Architectures},'' 2016.
    
    \bibitem{Kaya2011}
    Y.~Kaya, M.~Uyar, and R.~Tek$\backslash$D{\{}j{\}}n, ``{A Novel Crossover
      Operator for Genetic Algorithms: Ring Crossover},'' no.~May 2014, 2011.
    
    \end{thebibliography}
\end{document}
